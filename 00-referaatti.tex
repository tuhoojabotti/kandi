\noindent
JavaScript on ohjelmointikieli, joka suunniteltiin ensisijaisesti verkkosivujen tekijöille, joilla ei välttämättä ole paljon ohjelmointikokemusta. Sen idea oli täydentää Java-ohjelmointikieltä ja HTML-merkkauskieltä. Se mahdollistaa monimutkaisten Internet-selaimissa suoritettavien sovellusten kirjoittamisen~\cite{paolini1994netscape}.

Selainten suosio sovellusalustana on kasvanut ja samalla JavaScriptin käyttö on lisääntynyt räjähdysmäisesti. Se ei ole yllättävää, sillä kaikissa kuluttajatietokoneissa on jokin selain. Lisäksi sovellusten jakaminen on helppoa, koska siihen riittää pelkkä URL-osoite.

Selaimet ovat kehittyneet ja niiden käyttämiä teknologioita on standardoitu. Tämä on auttanut tekemään JavaScriptistä varteenotettavan vaihtoehdon moderniin sovelluskehitykseen. JavaScriptin käyttö ei kuitenkaan rajoitu vain selaimiin, vaan sitä käytetään myös palvelinsovelluksissa sekä muissa ilman selainta toimivissa sovelluksissa, kuten esimerkiksi Atom-tekstieditorissa~\cite{atom}.

JavaScript on dynaaminen oliopohjainen kieli, mutta se tukee myös imperatiivista ja funktionaalista ohjelmointityyliä. JavaScript siis tarjoaa monta tapaa toteuttaa samoja asioita~\cite[Osio 4.2.1.]{es6}. Muuttujat JavaScriptissä ovat dynaamisesti tyypitettyjä ja tämä helpottaa ohjelmointia tekemällä muuttujien käytöstä vapaampaa. Tästä dynaamisuudesta seuraa kuitenkin myös ongelmia, sillä virtuaalikoneiden on vaikea ennustaa dynaamisia muutoksia ja tehdä järkeviä optimointeja, joilla suorituskykyä saisi parannettua~\cite{Ahn2014}.

Kieltä kuitenkin kehitetään jatkuvasti ja siihen ollaan tuomassa muun muassa luokka- ja moduulijärjestelmät~\cite[Osiot~14.5.~ja~15.2.]{es6}, jotka toivottavasti yhtenäistävät toteutustapoja ja mahdollistavat tällä tavoin paremman ennustettavuuden ja suorituskyvyn.

Modernit JavaScript-virtuaalikoneet ovat kehittyneet paljon viime vuosina ja niihin on toteutettu monimutkaisia ja kekseliäitä menetelmiä suorituskyvyn parantamiseksi. Esimerkkejä tälläisistä ovat muun muassa piiloluokat (hidden classes) ja erilaiset ajonaikaiset optimoivat kääntäjät.

Suuri osa virtuaalikoneiden parannuksista perustuvat oletukseen, että vaikka kieli itsessään on dynaaminen, ohjelmat kuitenkin käyttäytyvät yleensä melko staattisesti. Keräämällä tyyppitietoa suorituksen aikana, virtuaalikoneet pystyvät esimerkiksi luomaan optimoitua konekoodia osalle lähdekoodista.

On selvää, että JavaScriptin standardoiminen, sen käytön lisääntyminen ja selainvalmistajien keskinäinen kilpailu suorituskyvystä on parantanut kielen asemaa ja mainetta. JavaScriptin rooli on muuttunut skriptikielestä yleiskäyttöiseksi ohjelmointikieleksi~\cite[Osio~4.]{es6}. Kielen tulevaisuus näyttää lupaavalta. Siihen on tulossa paljon ohjelmointia helpottavia ominaisuuksia ja tapoja välttää yleisiä sudenkuoppia, joihin varsinkin aloittelevat ohjelmoijat usein törmäävät.