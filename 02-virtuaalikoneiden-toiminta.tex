\section{Virtuaalikoneiden toiminta}

Virtuaalikone on ohjelma, joka tarjoaa todellisen tai hypoteettisen laitteen toiminnallisuuksia muille ohjelmille hyödyntäen sitä suorittavan \textit{isäntäjärjestelmän} abstraktioita ja palveluita. Virtuaalikone voi virtualisoida esimerkiksi CD-asemaa, käyttämällä isännän tiedostojärjestelmää hyväksi, jolloin virtuaalikoneessa suoritettava ohjelma luulee lukevansa CD-levyä, kun todellisuudessa tieto tulee kiintolevyltä.

\subsection{Kahdenlaisia virtuaalikoneita}

Virtuaalikoneita on kahdenlaisia, \textit{järjestelmä-} ja \textit{prosessivirtuaalikoneita}~\cite[s.~33]{vms}. Järjestelmävirtuaalikone tarjoaa kokonaisen käyttöjärjestelmän palvelut toisin kuin prosessivirtuaalikone, joka tarjoaa vaan yhden prosessin suorittamista varten tarvittavat palvelut. Tässä tutkielmassa virtuaalikoneella tarkoitetaan JavaScript-ohjelmia suorittavaa prosessivirtuaalikonetta.

Yksi suurimmista virtuaalikoneiden hyödyistä on, että ohjelma tarvitsee kääntää usean alustan sijaan yhdelle virtuaalikoneelle. Tästä seuraa, että ohjelma toimii kaikilla niillä alustoilla, joille kyseinen virtuaalikone on toteutettu. Virtuaalikoneessa suoritettava ohjelma pääsee käsiksi vain virtuaalikoneen tarjoamiin palveluihin, jolloin ohjelmat on helpompi eristää isäntäkäyttöjärjestelmästä ja laitteistosta~\cite[s.~36]{vms}. Pahantahtoisen ohjelman on siis löydettävä haavoittuvuus sekä virtuaalikoneesta että isännästä voidakseen aiheuttaa ongelmia.

\subsection{Perinteisen JavaScript-virtuaalikoneen anatomia}

Ensimmäisen JavaScript-virtuaalikoneen nimi on SpiderMonkey~\cite{spidermonkey}. Se toteutettiin Netscape-selainta varten vuonna 1995. Nykyään sitä ylläpitää Mozilla ja sitä käytetään muun muassa Mozillan Firefox-selaimessa. Nykyinen SpiderMonkey on kehittynyt paljon, mutta se koostuu yhä kolmesta fundamentaalisesta komponentista: \textit{kääntäjä}, \textit{tulkki} ja \textit{roskienkerääjä}~\cite{spidermonkeydesign}. Tämän arkkitehtuurin lisäksi on toteutettu myös muita menetelmiä suorituskyvyn parantamiseksi.

SpiderMonkeyn kääntäjä huolehtii koodin \textit{jäsentämisestä} (parsing) ja kääntämisestä \textit{tavukoodimuotoiseksi}. Virtuaalikoneen tavukoodi on todellisen koneen konekoodin kaltaista, ja yksinkertaisista operaatioista muodostuvaa tavukoodia on helpompi suorittaa kuin alkuperäistä tekstimuotoista ohjelmakoodia. Jotkin virtuaalikoneet eivät käytä tavukoodiesitystä, vaan muodostavat koodista ainoastaan \textit{abstraktin syntaksipuun}. Abstrakti syntaksipuu on nimensä mukaisesti ohjelmakoodista muodostettu puumainen abstrakti esitystapa.

Kääntäminen suoritetaan ''laiskasti'' eli koko ohjelmaa ei käännetä heti, vaan koodilohkoja käännetään osissa sitä mukaa kun suoritus etenee. Esimerkiksi suuret kirjastot eivät siis kasvata käännösaikaa, sillä niistä käännetään vain ne osat, joita oikeasti käytetään.

Tulkin tehtävä on suorittaa ohjelmaa. Tulkki siis lukee tavukoodia käsky kerrallaan tai käy läpi abstraktia syntaksipuuta ja kutsuu tarvittavia palveluja isäntäjärjestelmästään. Tulkista on siis oltava oma versionsa jokaista eri alustaa varten. Yksi tulkin eduista on, että se on joustavampi kuin todellinen prosessori. Tulkkiin voi toteuttaa monipuolisempia toimintoja helpommin kuin todelliseen prosessoriin.

Roskienkerääjän tehtävä on poistaa muistista muuttujat ja oliot, joihin ohjelmassa ei enää viitata. Roskienkeruun ansiosta ohjelmoijan ei tarvitse vapauttaa muistia itse, vaan järjestelmä hoitaa muistinhallinnan automaattisesti. Automaattinen muistinhallinta vähentää virheiden määrää, kuten muistivuotoja, mutta ei poista kaikkia ongelmia. Esimerkiksi muistivuodot ovat silti mahdollisia, jos ohjelmoija huomaamattaan jättää viittauksia olioihin, joita ei enää käytä.

\subsection{Suoritusaikainen kääntäminen}

Riippumatta toteutustavasta, tulkki joutuu aina tekemään useita konekäskyjä suhteessa tavukoodikäskyjen määrään~\cite[s.~35]{vms}. Tämä johtuu siitä, että tulkki toimii ohjelmistotasolla, kun todellinen prosessori hyödyntää laitteistoa suoraan. Valitettavasti tämä tekee tulkeista hitaita, tai ainakin hitaampia kuin haluttaisiin.

Vuonna 2008 Google julkaisi uuden selaimen, Google Chromen, jonka oli tarkoitus parantaa verkkosovellusten käyttökokemusta~\cite{chromepress}. Googlen kiinnostus käyttökokemuksen ja ennen kaikkea suorituskyvyn parantamisesta on ymmärrettävää, sillä yhtiöllä on paljon verkkopalveluita, jotka hyötyvät hyvästä suorituskyvystä. Tästä syystä Google päätyi toteuttamaan Chrome-selaintaan varten oman virtuaalikoneen nimeltään V8.

Mielenkiintoisen V8:sta tekee se, että siinä ei ole lainkaan tulkkia. V8-virtuaalikone kääntää koodin suoraan isäntäjärjestelmän konekoodiksi. Kääntäminen tehdään SpiderMonkeyn tapaan laiskasti niin sanotulla \textit{lähtötilannekääntäjällä} (baseline compiler)~\cite{v8design}. Nopean kääntämisen saavuttamiseksi lähtötilannekääntäjä ei tee monimutkaisia optimointeja, koska se hidastaisi ohjelman käynnistystä. Selainten tapauksessa sivujen nopea lataaminen on kriittistä käyttökokemuksen kannalta.

V8:n innoittamana muut virtuaalikonetoteutukset ovat muuttaneet toimintaansa siten, että tulkkia käytetään vain suorituksen alkuvaiheessa ja ohjelma pyritään kääntämään konekoodiksi mahdollisimman nopeasti \textit{suoritusaikaisella kääntäjällä} eli \textit{JIT-kääntäjällä} (Just-In-Time compiler).

Varsinkin usein kutsutut funktiot, eli niin sanotut ''kuumat funktiot'', halutaan kääntää mahdollisimman optimoiduksi konekoodiksi. Tätä varten käytetään erillistä optimoivaa JIT-kääntäjää, joka on hitaampi kuin lähtötilannekääntäjä, mutta tuottaa suorituskykyisempää konekoodia.

% TODO: OSR

%Normaalisti JIT-kääntäjät korvaavat optimoidun funktion vasta seuraavalla kutsukerralla, mutta V8 käyttää tekniikkaa nimeltä \textit{pinon päältä korvaaminen} (on-stack replacement)~\cite{osr}. Tämä mahdollistaa esimerkiksi hitaan silmukan rungon korvaamisen optimoidulla versiolla kesken silmukan suorituksen.

\subsection{Optimoinnin ongelmat}

Dynaamisesti tyypitetyllä ohjelmointikielellä toteutetusta ohjelmasta generoitu konekoodi vaatii paljon tyyppitarkastuksia ja poikkeustapauksia muuttujien tyypeille. On kuitenkin tehty oletuksia ohjelmien käyttäytymisestä. Esimerkiksi oletetaan, että usein kutsuttuja funktiota kutsutaan usein samantyyppisillä parametreilla~\cite[s.~2]{jsanalysis}.

Virtuaalikoneiden ei kannata suoraan generoida optimoitua konekoodia JavaScript-ohjelmista, sillä niillä ei ole tietoa muuttujien tyypeistä. Prosessorin kannalta on hyvin tärkeää tietää tehdäänkö jokin operaatio kokonaisluvuille, liukuluvuille tai kenties merkkijonoille. Lisäksi ei ole järkevää käyttää paljon aikaa koodin optimointiin, jos kyseinen koodi suoritetaan vain kerran tai muutamia kertoja.

Virtuaalikoneet keräävät tietoa ohjelman käyttäytymisestä suorituksen aikana. Tiedon kerääminen hoidetaan yleensä tulkissa, mutta V8:n tapauksessa lähtötilannekääntäjä lisää generoituun konekoodiin käskyjä keräämään tietoa ohjelmassa esiintyvistä tyypeistä~\cite{v8compilers}. Tämän profiloinnin ansiosta on mahdollista tehdä parempia optimointipäätöksiä.

Avoimen lähdekoodin WebKit-projekti sisältää JavaScriptCore-nimisen virtuaalikoneen, jota käytetään esimerkiksi Applen Safari selaimessa. Java\-Script\-Core koostuu tulkista, yksinkertaisesta JIT-kääntäjästä sekä Googlen V8:n innoittamana optimoivasta JIT-kääntäjästä, jota sen kehittäjät kutsuvat nimellä \textit{DFG-JIT}. Lyhenne DFG tarkoittaa \textit{tietovuokaaviota} (Data Flow Graph) ja se on ohjelman suoritusaikaisen tyyppitiedon tallentava tietorakenne. Siis muiden virtuaalikoneiden tapaan JavaScriptCore kerää ensin tyyppitietoa ja generoi sen avulla optimoitua konekoodia~\cite{javascriptcore}.

Automaattinen muistinhallinta eli roskienkeruu on hyödyllinen toiminnallisuus ohjelmoijan kannalta, mutta sen toteuttaminen hyvin ei ole helppoa. Ohjelmalle varatun muistin loppuessa virtuaalikoneen täytyy pysäyttää ohjelman suoritus ja käydä läpi muistin sisältö vapauttaen muistialueita, joihin ei ole enää viittauksia. Selaimen tapauksessa tämä voi aiheuttaa sovelluksen hidastumista. Hidastuminen huonontaa käyttökokemusta etenkin interaktiivisissa sovelluksissa tai animaation aikana.
