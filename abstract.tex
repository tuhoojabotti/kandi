JavaScript-virtuaalikoneet ovat perinteisesti toimineet tulkkaamalla lähdekoodista muodostettua abstraktia syntaksipuuta tai tavukoodia. Virtuaalikoneissa on otettu käyttöön monenlaisia JIT-kääntäjiä, mutta tulkkia käytetään yhä varsinkin suorituksen alussa sen keveyden ja nopeuden takia. JavaScriptin dynaamisuudesta johtuen virtuaalikoneiden kehittäjät ovat joutuneet toteuttamaan monimutkaisia menetelmiä tehokkaan konekoodin tuottamiseksi JavaScript-ohjelmista.

Monet optimoinnit käyttävät hyväksi oletusta, että sovellukset käyttäytyvät melko staattisesti, kielen dynaamisuudesta huolimatta. Tämän oletuksen nojalla on pystytty hyödyntämään optimointimenetelmiä, joita tyypillisesti käytetään staattisesti tyypitettyjen kielten kanssa.

Oletus staattisesta käytöksestä on havaittu ongelmalliseksi. Vaikka yleisesti käytössä olevat suorituskykytestit käyttäytyvät varsin staattisesti, todelliset sovellukset hyödyntävät kielen dynaamisuutta enemmän, mikä heikentää optimointimenetelmien toimivuutta.