JavaScript-virtuaalikoneet ovat perinteisesti toimineet kääntämällä ohjelman tavukoodiksi ja tulkkaamalla sitä. Modernit virtuaalikoneet ovat ottaneet käyttöön erilaisia suoritusaikaisia kääntäjiä eli JIT-kääntäjiä tai jättäneet tulkin kokonaan pois. JavaScriptin dynaamisuuden takia virtuaalikoneiden kehittäjät ovat joutuneet toteuttamaan monimutkaisia menetelmiä tehokkaan konekoodin tuottamiseksi käyttäen hyväksi oletusta, että sovellukset käyttäytyvät melko staattisesti, dynaamisesta kielestä huolimatta.

Automaattista muistinhallintaa on jouduttu parantamaan sovellusten vaatimusten kasvaessa. Selaimet eivät voi pysäyttää ohjelman suoritusta useaksi sadaksi millisekunniksi roskienkeräyksen takia, sillä tämä huonontaa käyttökokemusta varsinkin interaktiivisissa sovelluksissa.