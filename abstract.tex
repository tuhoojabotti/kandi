JavaScript-virtuaalikoneet ovat perinteisesti toimineet tulkkaamalla lähdekoodista muodostettua abstraktia syntaksipuuta tai käännettyä tavukoodia. Virtuaalikoneissa on otettu käyttöön monenlaisia JIT-kääntäjiä ja joissain tapauksissa tulkki on jätetty kokonaan pois. JavaScriptin dynaamisuuden takia virtuaalikoneiden kehittäjät ovat joutuneet toteuttamaan monimutkaisia menetelmiä tehokkaan konekoodin tuottamiseksi.

Monet optimoinnit käyttävät hyväksi oletusta, että sovellukset käyttäytyvät melko staattisesti, kielen dynaamisuudesta huolimatta. Tämän oletuksen nojalla on pystytty hyödyntämään optimointimenetelmiä, joita tyypillisesti käytetään staattisesti tyypitettyjen kielten kanssa.

Tutkimuksissa on kuitenkin havaittu, että oletus staattisesta käytöksestä saattaa olla virheellinen. Vaikka yleisesti käytössä olevat suorituskykytestit käyttäytyvät varsin staattisesti, todelliset sovellukset hyödyntävät kielen dynaamisuutta enemmän, mikä vähentää optimoinneista saatavia hyötyjä.

%Automaattista muistinhallintaa on jouduttu parantamaan sovellusten vaatimusten kasvaessa. Selaimet eivät voi pysäyttää ohjelman suoritusta useaksi sadaksi millisekunniksi roskienkeräyksen takia, sillä tämä huonontaa käyttökokemusta varsinkin interaktiivisissa sovelluksissa.