% --- Template for thesis / report with tktltiki2 class ---
% 
% last updated 2013/02/15 for tkltiki2 v1.02

\documentclass[finnish]{tktltiki2}

% tktltiki2 automatically loads babel, so you can simply
% give the language parameter (e.g. finnish, swedish, english, british) as
% a parameter for the class: \documentclass[finnish]{tktltiki2}.
% The information on title and abstract is generated automatically depending on
% the language, see below if you need to change any of these manually.
% 
% Class options:
% - grading                 -- Print labels for grading information on the front page.
% - disablelastpagecounter  -- Disables the automatic generation of page number information
%                              in the abstract. See also \numberofpagesinformation{} command below.
%
% The class also respects the following options of article class:
%   10pt, 11pt, 12pt, final, draft, oneside, twoside,
%   openright, openany, onecolumn, twocolumn, leqno, fleqn
%
% The default font size is 11pt. The paper size used is A4, other sizes are not supported.
%
% rubber: module pdftex

% --- General packages ---

\usepackage[utf8]{inputenc}
\usepackage[T1]{fontenc}
\usepackage{lmodern}
\usepackage{microtype}
\usepackage{amsfonts,amsmath,amssymb,amsthm,booktabs,color,enumitem,graphicx}
\usepackage[pdftex,hidelinks]{hyperref}

% Automatically set the PDF metadata fields
\makeatletter
\AtBeginDocument{\hypersetup{pdftitle = {\@title}, pdfauthor = {\@author}}}
\makeatother

% babelbib for non-english bibliography using bibtex
\usepackage[fixlanguage]{babelbib}
\selectbiblanguage{finnish}

% Babelbib doesn't support finnish ordinals for example edition = 6 -> kuudes painos.
\declarebtxcommands{finnish}{
  \def\btxnumeralshort#1{
    #1.}
  \def\btxnumerallong#1{
    \ifnumber{#1}{
      \ifcase#1 nollas\or ensimmäinen\or toinen\or kolmas\or neljäs\or viides\or
        kuudes\or seitsemäs\or kahdeksas\or yhdeksäs\or kymmenes\else
        #1.
      \fi
    }{#1}}
}

% Remove [brackets] around keys.
\makeatletter
\renewcommand\@biblabel[1]{\hfill #1.}
\makeatother

% Try not to break small words. (että, jotta, etc.) (range: 0 - 10 000)
\pretolerance=1000

% add bibliography to the table of contents
\usepackage[nottoc]{tocbibind}
% tocbibind renames the bibliography, use the following to change it back
\settocbibname{Lähteet}

% --- Theorem environment definitions ---

\newtheorem{lau}{Lause}
\newtheorem{lem}[lau]{Lemma}
\newtheorem{kor}[lau]{Korollaari}

\theoremstyle{definition}
\newtheorem{maar}[lau]{Määritelmä}
\newtheorem{ong}{Ongelma}
\newtheorem{alg}[lau]{Algoritmi}
\newtheorem{esim}[lau]{Esimerkki}

\theoremstyle{remark}
\newtheorem*{huom}{Huomautus}

% --- tktltiki2 options ---

\title{JavaScript ja virtuaalikoneet}
\author{Ville Lahdenvuo}
\date{\today}
\level{Aine}
%\level{Kandidaatintutkielma}
\abstract{Tiivistelmä.}

% The following can be used to specify keywords and classification of the paper:

\keywords{JavaScript, virtuaalikone, suorituskyky}

% classification according to ACM Computing Classification System (http://www.acm.org/about/class/)
\classification{
  \textbf{Software and its engineering $\rightarrow$ Virtual machines} \\
  Software and its engineering $\rightarrow$ Very high level languages
}

% If the automatic page number counting is not working as desired in your case,
% uncomment the following to manually set the number of pages displayed in the abstract page:
%
% \numberofpagesinformation{16 sivua + 10 sivua liitteissä}

\begin{document}

% --- Front matter ---

\frontmatter      % roman page numbering for front matter

\maketitle        % title page
\makeabstract     % abstract page

\tableofcontents  % table of contents

% --- Main matter ---

\mainmatter       % clear page, start arabic page numbering

% Set bigger space between lines.
\renewcommand{\baselinestretch}{1.5}
\selectfont

%\noindent
JavaScript on ohjelmointikieli, joka suunniteltiin ensisijaisesti verkkosivujen tekijöille, joilla ei välttämättä ole paljon ohjelmointikokemusta. Sen idea oli täydentää Java-ohjelmointikieltä ja HTML-merkkauskieltä. Se mahdollistaa monimutkaisten Internet-selaimissa suoritettavien sovellusten kirjoittamisen ilman selainlaajennuksia~\cite{paolini1994netscape}.

Selainten suosio sovellusalustana on kasvanut ja samalla JavaScriptin käyttö on lisääntynyt räjähdysmäisesti. Se ei ole yllättävää, sillä kaikissa kuluttajatietokoneissa on jokin selain. Lisäksi sovellusten jakaminen on helppoa, koska siihen riittää pelkkä URL-osoite.

Selaimet ovat kehittyneet ja niiden käyttämiä teknologioita on standardoitu. Tämä on auttanut tekemään JavaScriptistä varteenotettavan vaihtoehdon moderniin sovelluskehitykseen. JavaScriptin käyttö ei kuitenkaan rajoitu vain selaimiin, vaan sitä käytetään myös palvelinsovelluksissa sekä muissa ilman selainta toimivissa sovelluksissa, kuten esimerkiksi Atom-tekstieditorissa~\cite{atom}.

JavaScript on dynaaminen oliopohjainen kieli, mutta se tukee myös imperatiivista ja funktionaalista ohjelmointityyliä. JavaScript siis tarjoaa monta tapaa toteuttaa samoja asioita~\cite[Osio 4.2.1.]{es6}. Muuttujat JavaScriptissä ovat dynaamisesti tyypitettyjä ja tämä helpottaa ohjelmointia tekemällä muuttujien käytöstä vapaampaa. Tästä dynaamisuudesta seuraa kuitenkin myös ongelmia, sillä virtuaalikoneiden on vaikea ennustaa dynaamisia muutoksia ja tehdä järkeviä optimointeja, joilla suorituskykyä saisi parannettua~\cite{Ahn2014}.

Modernit JavaScript-virtuaalikoneet ovat kehittyneet paljon viime vuosina ja niihin on toteutettu monimutkaisia ja kekseliäitä menetelmiä suorituskyvyn parantamiseksi. Esimerkkejä tälläisistä ovat muun muassa piiloluokat (hidden classes) ja erilaiset ajonaikaiset optimoivat kääntäjät.

Suuri osa virtuaalikoneiden parannuksista perustuvat oletukseen, että vaikka kieli itsessään on dynaaminen, ohjelmat kuitenkin käyttäytyvät yleensä melko staattisesti. Keräämällä tyyppitietoa suorituksen aikana, virtuaalikoneet pystyvät esimerkiksi luomaan optimoitua konekoodia osalle lähdekoodista. Tämä edellyttää tietenkin, että JavaScript-ohjelmoija tietää miten asiat kannattaa toteuttaa.

Kieltä kuitenkin kehitetään jatkuvasti ja siihen ollaan tuomassa muun muassa luokka- ja moduulijärjestelmät~\cite[Osiot~14.5.~ja~15.2.]{es6}, jotka pyrkivät yhtenäistämään toteutustapoja ja mahdollistavat tällä tavoin paremman ennustettavuuden ja suorituskyvyn.

On selvää, että JavaScriptin standardoiminen, sen käytön lisääntyminen ja selainvalmistajien keskinäinen kilpailu suorituskyvystä on parantanut kielen asemaa ja mainetta. JavaScriptin rooli on muuttunut skriptikielestä yleiskäyttöiseksi ohjelmointikieleksi~\cite[Osio~4.]{es6}. Kielen tulevaisuus näyttää lupaavalta. Siihen on tulossa paljon ohjelmointia helpottavia ominaisuuksia ja tapoja välttää yleisiä sudenkuoppia, joihin varsinkin aloittelevat ohjelmoijat usein törmäävät.
\section{Johdanto}

JavaScript on ohjelmointikieli, joka suunniteltiin ensisijaisesti verkkosivujen tekijöille. Sen tavoite oli täydentää Java-ohjelmointikieltä ja HTML-merkkauskieltä. JavaScript mahdollistaa monimutkaisten Internet-selaimissa suoritettavien sovellusten kirjoittamisen ilman selainlaajennuksia~\cite{paolini1994netscape}.

Selainten suosio sovellusalustana on kasvanut ja samalla JavaScriptin käyttö on lisääntynyt paljon. Kasvua siivittää se, että kaikissa kuluttajatietokoneissa on jokin selain ja sovellusten käyttämiseen riittää verkkosivuilla vieraileminen. Käyttäjän ei tarvitse asentaa mitään ennen sovelluksen käyttöä.

Selaimet ovat kehittyneet ja niiden käyttämiä teknologioita on standardisoitu. Näistä niin sanotuista Web-teknologioista, joihin JavaScript lasketaan, on tullut varteenotettava vaihtoehto moderniin sovelluskehitykseen. Web-teknologioiden käyttö ei kuitenkaan rajoitu vain selaimiin. Niillä on toteutettu esimerkiksi palvelinsovelluksia sekä kokonaan ilman selainta toimivia sovelluksia, kuten Atom-tekstieditori~\cite{atom}.

JavaScript on dynaaminen oliopohjainen kieli, mutta se tukee myös imperatiivista ja funktionaalista ohjelmointityyliä. JavaScript tarjoaa siis monia tapoja toteuttaa sama asia~\cite[4.2.1.]{es6}. Muuttujat JavaScriptissä ovat dynaamisesti tyypitettyjä ja tämä helpottaa ohjelmakoodin kirjoittamista. Dynaamisuudesta seuraa kuitenkin myös ongelmia, sillä virtuaalikoneiden on vaikea ennustaa dynaamisia muutoksia ja tehdä järkeviä optimointeja, joilla suorituskykyä voitaisiin parantaa~\cite[s.~497]{Ahn2014}.

Modernit JavaScript-virtuaalikoneet ovat kehittyneet paljon viime vuosina. Niihin on toteutettu monimutkaisia ja kekseliäitä menetelmiä suorituskyvyn parantamiseksi. Suuri osa virtuaalikoneiden optimoinneista perustuu oletukseen, että dynaamisuudesta huolimatta ohjelmat käyttäytyvät suorituksen aikana yleensä melko staattisesti. Keräämällä tyyppitietoa suorituksen aikana, virtuaalikoneet pystyvät esimerkiksi luomaan optimoitua konekoodia osalle lähdekoodista. Optimoitavuus edellyttää, että ohjelmoija tietää miten asiat kannattaa toteuttaa. Jos hyödyntää dynaamisuutta liikaa, voi helposti tehdä koodia, jota virtuaalikone ei pysty optimoimaan.

%Esimerkkejä tälläisistä ovat muun muassa \textit{piiloluokat} (hidden classes)~\cite{v8design} ja erilaiset suorituksenaikaiset optimoivat kääntäjät. Piiloluokat ovat staattisia luokkia, joita virtuaalikone käyttää objektien kuvaamiseen. Jos objektia muutetaan, esimerkiksi lisäämällä uusi kenttä, luodaan sille uusi piiloluokka. Piiloluokkien hyödyt tulevat esiin, jos ohjelmassa on paljon samanlaisia objekteja, jotka voivat käyttää samaa piiloluokkaa.

JavaScriptiä kuitenkin kehitetään jatkuvasti ja siihen on tuotu muun muassa luokka- ja moduulijärjestelmät~\cite[14.5.~ja~15.2.]{es6}. Nämä auttavat yhtenäistämään erilaisia toteutustapoja ja mahdollistavat tällä tavoin aikaisempaa paremmin ennustettavan käytöksen. Kun käytetään luokkasyntaksia, tulee käytettyä yhtenäistä tapaa muodostaa objekteja. Ennustettavuudesta seuraa parempi optimoitavuus ja suorituskyky~\cite[s.~497]{Ahn2014}.

JavaScriptin standardisoiminen, sen käytön lisääntyminen ja selainvalmistajien keskinäinen kilpailu suorituskyvystä on parantanut kielen asemaa ja mainetta. JavaScriptin rooli on muuttunut skriptikielestä yleiskäyttöiseksi ohjelmointikieleksi~\cite[4.]{es6}. Kielen tulevaisuus näyttää lupaavalta. Siihen on tullut paljon ohjelmointia helpottavia ominaisuuksia ja tapoja välttää yleisiä sudenkuoppia, joihin varsinkin aloittelevat ohjelmoijat usein törmäävät.
\section{Virtuaalikoneiden toiminta}

Virtuaalikone on ohjelma, joka tarjoaa todellisen tai hypoteettisen laitteen toiminnallisuuksia muille ohjelmille hyödyntäen sitä suorittavan \textit{isäntäjärjestelmän} abstraktioita. Virtuaalikone voi esimerkiksi virtualisoida optista asemaa käyttämällä isännän tiedostojärjestelmää hyväksi, jolloin virtuaalikoneessa suoritettava ohjelma luulee lukevansa optista levyä, kun todellisuudessa tieto tulee kiintolevyltä. Virtuaalikoneita on kahdenlaisia, \textit{järjestelmä-} ja \textit{prosessivirtuaalikoneita}~\cite[s.~33]{vms}. Järjestelmävirtuaalikone tarjoaa kokonaisen käyttöjärjestelmän palvelut toisin kuin prosessivirtuaalikone, joka tarjoaa vaan yhden prosessin suorittamista varten tarvittavat palvelut.

\subsection{Virtuaalikoneen anatomia}

Tässä tutkielmassa virtuaalikoneella tarkoitetaan korkean tason ohjelmointikielellä toteutetun ohjelman suorittavaa virtuaalikonetta. Tällaiset virtuaalikoneet ovat prosessivirtuaalikoneita. Ohjelma riittää kääntää virtuaalikoneelle, jolloin se toimii kaikilla alustoilla, joille kyseinen virtuaalikone on toteutettu.

Virtuaalikonemalli parantaa myös tietoturvaa. Virtuaalikoneessa suoritettava ohjelma pääsee käsiksi vain virtuaalikoneen tarjoamiin palveluihin, jolloin ohjelmat on helpompi eristää käyttöjärjestelmästä ja laitteistosta~\cite[s.~36]{vms}. Pahan tahtoisen ohjelman on siis löydettävä haavoittuvuus sekä virtuaalikoneesta että sen isännästä.

Ensimmäisen JavaScript-virtuaalikoneen nimi on SpiderMonkey~\cite{spidermonkey}. Se toteutettiin Netscape-selainta varten vuonna 1995 ja nykyään sitä ylläpitää Mozilla ja sitä käytetään muun muassa Mozillan Firefox-selaimessa. Nykyinen SpiderMonkey koostuu kolmesta fundamentaalisesta komponentista: \textit{kääntäjä}, \textit{tulkki} ja \textit{roskienkerääjä}~\cite{spidermonkeydesign}.

Kääntäjä huolehtii koodin \textit{jäsentämisestä} (parsing) ja kääntämisestä \textit{tavukoodiksi}. Virtuaalikoneen tavukoodi on verrattavissa todellisen koneen konekoodiin. Sitä on helpompi käsitellä ohjelmallisesti kuin tekstimuotoista ohjelmakoodia, sillä se on yksinkertaisempaa syntaksiltaan, joskin runsassanaisempaa.

Tulkin tehtävä on suorittaa tavukoodia. Tulkki siis lukee tavukoodia ja tekee tarvittavat toiminnot, jotka riippuvat alustasta. Tulkista on siis oltava oma versionsa jokaista tuettua alustaa varten. Todellisuudessa tulkkia käytetään vain suorituksen alkuvaiheessa keräämään tyyppitietoa. Usein kutsutut, niin sanotut ``kuumat'' funktiot, pyritään kääntämään optimoiduksi konekoodiksi \textit{suorituksenaikaisella kääntäjällä} eli \textit{JIT-kääntäjällä} (Just-In-Time compiler)~[lähde?].

Roskienkerääjän tehtävä on yksinkertaisesti poistaa muistista muuttujat ja oliot, joihin ohjelmassa ei enää viitata. Roskienkerääjän ansiosta ohjelmoijan ei tarvitse vapauttaa muistia itse, vaan järjestelmä hoitaa muistinhallinnan automaattisesti. Automaattinen muistinhallinta vähentää virheiden määrää, kuten muistivuotoja, mutta ei poista kaikkia ongelmia~[lähde?].

\subsection{Esimerkki: V8}

Vuonna 2008 Google julkaisi uuden selaimen, Google Chromen, jonka oli tarkoitus parantaa verkkosovellusten käyttökokemusta~\cite{chromepress}. Googlen kiinnostus käyttökokemuksen parantamisesta on ymmärrettävää, sillä yhtiöllä on paljon verkkopalveluita, jotka hyötyvät hyvästä suorituskyvystä. Näistä syistä Google päätyi toteuttamaan oman JavaScript-virtuaalikoneen V8:n.

Mielenkiintoisen V8:sta tekee se, että siinä ei ole lainkaan tulkkia. Sen sijaan V8 kääntää JavaScript koodin suoraan konekoodiksi ennen suorittamista. SpiderMonkeyn tapaan se kerää tyyppitietoa suorituksen aikana ja käyttää optimoivaa JIT-kääntäjää suorituskyvyn parantamiseksi~\cite{v8compilers}.

%V8:ssa on luonnollisesti myös roskienkerääjä, sillä kielen standardiin kuuluu automaattinen muistinhallinta. Muistinhallinnasta kerrotaan lisää sen suorituskykyä käsittelevässä kappaleessa.

\subsection{Esimerkki: JavaScriptCore}

JavaScriptCore on WebKit-nimisen Web-teknologioita toteuttavan ohjelmistokomponentin JavaScript-virtuaalikone. Apple käyttää WebKitiä ja \mbox{JavaScriptCorea} Safari-selaimessaan.

JavaScriptCore koostuu tulkista, yksinkertaisesta JIT-kääntäjästä sekä Googlen V8:n innoittamana optimoivasta JIT-kääntäjästä, jota he kutsuvat nimellä DFG-JIT. DFG tulee sanoista \textit{Data Flow Graph}, joka kuvaa ohjelman suorituksenaikaisen tyyppitiedon tallentavaa tietorakennetta. Eli kuten aikaisempien esimerkkien kohdalla, virtuaalikone ensin kerää tyyppitietoa ja sitten generoi optimoitua konekoodia~\cite{javascriptcore}.

\subsection{Esimerkki: Chakra}

Tekstiä tähän...

\subsection{JavaScriptin tuomat haasteet}

Kuten aikaisemmista esimerkeistä käy ilmi, virtuaalikoneet eivät voi suoraan generoida optimoitua konekoodia JavaScript-ohjelmista, sillä ei ole tietoa muuttujien tyypeistä. Prosessorin kannalta on hyvin tärkeää tietää tehdäänkö jokin operaatio kokonaisluvuille, liukuluvuille vai kenties merkkijonoille.

Automaattinen roskienkeruu on todella kätevä toiminnallisuus ohjelmoijan kannalta, mutta sen toteuttaminen hyvin on haastavaa. Ohjelman on pysähdyttävä ja käytävä läpi muistin sisältöä ja vapauttaa muistialueita, jotka eivät ole enää käytössä. Selaimen tapauksessa tämä voi aiheuttaa verkkosovelluksen hidastumista ja huonontaa käyttökokemusta, varsinkin jos kyseessä on interaktiivinen toiminto tai animaatio.
\section{Virtuaalikoneiden suorituskyky}

% --- References ---

\bibliographystyle{babplain-lf}
%\bibliographystyle{babalpha-lf}
\pagebreak
\bibliography{references}


% --- Appendices ---

% uncomment the following

% \newpage
% \appendix
% 
% \section{Esimerkkiliite}

\end{document}