% --- Template for thesis / report with tktltiki2 class ---
% 
% last updated 2013/02/15 for tkltiki2 v1.02

\documentclass[finnish]{tktltiki2}

% tktltiki2 automatically loads babel, so you can simply
% give the language parameter (e.g. finnish, swedish, english, british) as
% a parameter for the class: \documentclass[finnish]{tktltiki2}.
% The information on title and abstract is generated automatically depending on
% the language, see below if you need to change any of these manually.
% 
% Class options:
% - grading                 -- Print labels for grading information on the front page.
% - disablelastpagecounter  -- Disables the automatic generation of page number information
%                              in the abstract. See also \numberofpagesinformation{} command below.
%
% The class also respects the following options of article class:
%   10pt, 11pt, 12pt, final, draft, oneside, twoside,
%   openright, openany, onecolumn, twocolumn, leqno, fleqn
%
% The default font size is 11pt. The paper size used is A4, other sizes are not supported.
%
% rubber: module pdftex

% --- General packages ---

\usepackage[utf8]{inputenc}
\usepackage[T1]{fontenc}
\usepackage{lmodern}
\usepackage{microtype}
\usepackage{amsfonts,amsmath,amssymb,amsthm,booktabs,color,enumitem,graphicx}
\usepackage[pdftex,hidelinks]{hyperref}

% Automatically set the PDF metadata fields
\makeatletter
\AtBeginDocument{\hypersetup{pdftitle = {\@title}, pdfauthor = {\@author}}}
\makeatother

% --- Language-related settings ---
%
% these should be modified according to your language

% babelbib for non-english bibliography using bibtex
\usepackage[fixlanguage]{babelbib}
\selectbiblanguage{finnish}

% Babelbib doesn't support finnish ordinals for example edition = 6 -> kuudes painos.
\declarebtxcommands{finnish}{
  \def\btxnumeralshort#1{
    #1.}
  \def\btxnumerallong#1{
    \ifnumber{#1}{
      \ifcase#1 nollas\or ensimmäinen\or toinen\or kolmas\or neljäs\or viides\or
        kuudes\or seitsemäs\or kahdeksas\or yhdeksäs\or kymmenes\else
        #1.
      \fi
    }{#1}}
}

% Remove [brackets] around keys.
\makeatletter
\renewcommand\@biblabel[1]{\hfill #1}
\makeatother

% add bibliography to the table of contents
\usepackage[nottoc]{tocbibind}
% tocbibind renames the bibliography, use the following to change it back
\settocbibname{Lähteet}

% --- Theorem environment definitions ---

\newtheorem{lau}{Lause}
\newtheorem{lem}[lau]{Lemma}
\newtheorem{kor}[lau]{Korollaari}

\theoremstyle{definition}
\newtheorem{maar}[lau]{Määritelmä}
\newtheorem{ong}{Ongelma}
\newtheorem{alg}[lau]{Algoritmi}
\newtheorem{esim}[lau]{Esimerkki}

\theoremstyle{remark}
\newtheorem*{huom}{Huomautus}


% --- tktltiki2 options ---
%
% The following commands define the information used to generate title and
% abstract pages. The following entries should be always specified:

\title{JavaScript ja virtuaalikoneet}
\author{Ville Lahdenvuo}
\date{\today}
\level{Referaatti}
%\level{Kandidaatintutkielma}
\level{Referaatti}
\abstract{Tiivistelmä.}

% The following can be used to specify keywords and classification of the paper:

\keywords{avainsana 1, avainsana 2, avainsana 3}

% classification according to ACM Computing Classification System (http://www.acm.org/about/class/)
% This is probably mostly relevant for computer scientists
% uncomment the following; contents of \classification will be printed under the abstract with a title
% "ACM Computing Classification System (CCS):"
% \classification{}

% If the automatic page number counting is not working as desired in your case,
% uncomment the following to manually set the number of pages displayed in the abstract page:
%
% \numberofpagesinformation{16 sivua + 10 sivua liitteissä}
%
% If you are not a computer scientist, you will want to uncomment the following by hand and specify
% your department, faculty and subject by hand:
%
% \faculty{Matemaattis-luonnontieteellinen}
% \department{Tietojenkäsittelytieteen laitos}
% \subject{Tietojenkäsittelytiede}
%
% If you are not from the University of Helsinki, then you will most likely want to set these also:
%
% \university{Helsingin Yliopisto}
% \universitylong{HELSINGIN YLIOPISTO --- HELSINGFORS UNIVERSITET --- UNIVERSITY OF HELSINKI} % displayed on the top of the abstract page
% \city{Helsinki}
%

\begin{document}

% --- Front matter ---

\frontmatter      % roman page numbering for front matter

\maketitle        % title page
%\makeabstract     % abstract page

%\tableofcontents  % table of contents

% --- Main matter ---

\mainmatter       % clear page, start arabic page numbering

% Set bigger space between lines.
\renewcommand{\baselinestretch}{1.5}
\selectfont

\noindent
JavaScript on ohjelmointikieli, joka suunniteltiin ensisijaisesti verkkosivujen tekijöille, joilla ei välttämättä ole paljon ohjelmointikokemusta. Sen idea oli täydentää Java-ohjelmointikieltä ja HTML-merkkauskieltä. Se mahdollistaa monimutkaisten Internet-selaimissa suoritettavien sovellusten kirjoittamisen ilman selainlaajennuksia~\cite{paolini1994netscape}.

Selainten suosio sovellusalustana on kasvanut ja samalla JavaScriptin käyttö on lisääntynyt räjähdysmäisesti. Se ei ole yllättävää, sillä kaikissa kuluttajatietokoneissa on jokin selain. Lisäksi sovellusten jakaminen on helppoa, koska siihen riittää pelkkä URL-osoite.

Selaimet ovat kehittyneet ja niiden käyttämiä teknologioita on standardoitu. Tämä on auttanut tekemään JavaScriptistä varteenotettavan vaihtoehdon moderniin sovelluskehitykseen. JavaScriptin käyttö ei kuitenkaan rajoitu vain selaimiin, vaan sitä käytetään myös palvelinsovelluksissa sekä muissa ilman selainta toimivissa sovelluksissa, kuten esimerkiksi Atom-tekstieditorissa~\cite{atom}.

JavaScript on dynaaminen oliopohjainen kieli, mutta se tukee myös imperatiivista ja funktionaalista ohjelmointityyliä. JavaScript siis tarjoaa monta tapaa toteuttaa samoja asioita~\cite[Osio 4.2.1.]{es6}. Muuttujat JavaScriptissä ovat dynaamisesti tyypitettyjä ja tämä helpottaa ohjelmointia tekemällä muuttujien käytöstä vapaampaa. Tästä dynaamisuudesta seuraa kuitenkin myös ongelmia, sillä virtuaalikoneiden on vaikea ennustaa dynaamisia muutoksia ja tehdä järkeviä optimointeja, joilla suorituskykyä saisi parannettua~\cite{Ahn2014}.

Modernit JavaScript-virtuaalikoneet ovat kehittyneet paljon viime vuosina ja niihin on toteutettu monimutkaisia ja kekseliäitä menetelmiä suorituskyvyn parantamiseksi. Esimerkkejä tälläisistä ovat muun muassa piiloluokat (hidden classes) ja erilaiset ajonaikaiset optimoivat kääntäjät.

Suuri osa virtuaalikoneiden parannuksista perustuvat oletukseen, että vaikka kieli itsessään on dynaaminen, ohjelmat kuitenkin käyttäytyvät yleensä melko staattisesti. Keräämällä tyyppitietoa suorituksen aikana, virtuaalikoneet pystyvät esimerkiksi luomaan optimoitua konekoodia osalle lähdekoodista. Tämä edellyttää tietenkin, että JavaScript-ohjelmoija tietää miten asiat kannattaa toteuttaa.

Kieltä kuitenkin kehitetään jatkuvasti ja siihen ollaan tuomassa muun muassa luokka- ja moduulijärjestelmät~\cite[Osiot~14.5.~ja~15.2.]{es6}, jotka pyrkivät yhtenäistämään toteutustapoja ja mahdollistavat tällä tavoin paremman ennustettavuuden ja suorituskyvyn.

On selvää, että JavaScriptin standardoiminen, sen käytön lisääntyminen ja selainvalmistajien keskinäinen kilpailu suorituskyvystä on parantanut kielen asemaa ja mainetta. JavaScriptin rooli on muuttunut skriptikielestä yleiskäyttöiseksi ohjelmointikieleksi~\cite[Osio~4.]{es6}. Kielen tulevaisuus näyttää lupaavalta. Siihen on tulossa paljon ohjelmointia helpottavia ominaisuuksia ja tapoja välttää yleisiä sudenkuoppia, joihin varsinkin aloittelevat ohjelmoijat usein törmäävät.

% --- References ---
%
% bibtex is used to generate the bibliography. The babplain style
% will generate numeric references (e.g. [1]) appropriate for theoretical
% computer science. If you need alphanumeric references (e.g [Tur90]), use
%
% \bibliographystyle{babalpha-lf}
%
% instead.

%\bibliographystyle{babplain-lf}
\bibliographystyle{babalpha-lf}
\bibliography{references}


% --- Appendices ---

% uncomment the following

% \newpage
% \appendix
% 
% \section{Esimerkkiliite}

\end{document}