\section{Johdanto}

JavaScript on ohjelmointikieli, joka suunniteltiin ensisijaisesti verkkosivujen tekijöille. Sen tavoite oli täydentää Java-ohjelmointikieltä ja HTML-merkkauskieltä. JavaScript mahdollistaa monipuolisten selaimissa suoritettavien sovellusten kirjoittamisen ilman selainlaajennuksia~\cite{paolini1994netscape}.

Selainten suosio sovellusalustana on kasvanut ja sen ansiosta JavaScriptin käyttö on lisääntynyt. Kasvua siivittää se, että kaikissa kuluttajatietokoneissa on jokin selain ja sovellusten käyttämiseen riittää verkkosivuilla vieraileminen. Käyttäjän ei tarvitse asentaa ohjelmaa ennen sovelluksen käyttöä.

Selaimet ovat kehittyneet ja niiden käyttämiä teknologioita on standardisoitu. Näistä niin sanotuista Web-teknologioista, joihin JavaScript lasketaan, on tullut varteenotettava vaihtoehto moderniin sovelluskehitykseen. Web-teknologioiden käyttö ei kuitenkaan rajoitu vain selaimiin. Niillä on toteutettu palvelinsovelluksia sekä kokonaan ilman selainta toimivia sovelluksia, kuten esimerkiksi Atom-tekstieditori~\cite{atom}.

JavaScript on dynaaminen oliopohjainen kieli, mutta se tukee myös imperatiivista ja funktionaalista ohjelmointityyliä. JavaScript tarjoaa siis monia tapoja toteuttaa samoja toiminnallisuuksia~\cite[4.2.1.]{es6}. Muuttujat JavaScriptissä ovat dynaamisesti tyypitettyjä, mikä helpottaa ohjelmakoodin kirjoittamista. Dynaamisuudesta seuraa kuitenkin myös ongelmia, sillä virtuaalikoneiden on vaikea ennustaa dynaamisia muutoksia ja tehdä järkeviä optimointeja, joilla suorituskykyä voitaisiin parantaa~\cite[s.~497]{Ahn2014}.

Modernit JavaScript-virtuaalikoneet ovat kehittyneet paljon viime vuosina. Niihin on toteutettu monimutkaisia ja kekseliäitä menetelmiä suorituskyvyn parantamiseksi. Suuri osa virtuaalikoneiden optimoinneista perustuu oletukseen, että dynaamisuudesta huolimatta ohjelmat käyttäytyvät suorituksen aikana yleensä melko staattisesti. Keräämällä tyyppitietoa suorituksen aikana, virtuaalikoneet pystyvät esimerkiksi luomaan optimoitua konekoodia osalle lähdekoodista. Optimoitavuus edellyttää, että ohjelmoija tietää miten asiat kannattaa toteuttaa. Jos hyödyntää dynaamisuutta liikaa, voi helposti tehdä koodia, jota virtuaalikone ei pysty optimoimaan.

JavaScriptiä kuitenkin kehitetään jatkuvasti ja siihen on tuotu muun muassa luokka- ja moduulijärjestelmät~\cite[14.5.~ja~15.2.]{es6}. Nämä auttavat yhtenäistämään erilaisia toteutustapoja ja mahdollistavat tällä tavoin aikaisempaa paremmin ennustettavan käytöksen. Käyttämällä luokkasyntaksia, saavutetaan yhtenäisempi tapa muodostaa olioita. Ennustettavuudesta seuraa parempi optimoitavuus ja suorituskyky~\cite[s.~497]{Ahn2014}.

JavaScriptin standardisoiminen, sen käytön lisääntyminen ja selainvalmistajien keskinäinen kilpailu suorituskyvystä on parantanut kielen asemaa ja mainetta. JavaScriptin rooli on muuttunut skriptikielestä yleiskäyttöiseksi ohjelmointikieleksi~\cite[4.]{es6}. Kielen tulevaisuus näyttää lupaavalta. Siihen on tullut paljon ohjelmointia helpottavia ominaisuuksia ja tapoja välttää yleisiä sudenkuoppia, joihin varsinkin aloittelevat ohjelmoijat usein törmäävät.